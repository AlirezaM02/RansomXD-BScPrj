% Chapter 5
\chapter{بحث، پیشنهادات و مرور ادبیات}

\section{مرور ادبیات}
در این بخش، به بررسی کارهای مرتبط در حوزه تشخیص باج‌افزار پرداخته می‌شود. مطالعات پیشین نشان می‌دهد که:
\begin{itemize}
    \item استفاده از الگوریتم‌های یادگیری ماشین سنتی مانند رندوم فارست و SVM همراه با استخراج ویژگی‌های دستی، در تشخیص باج‌افزار عملکرد نسبتاً مناسبی داشته است \cite{ref1}.
    \item مدل‌های مبتنی بر XGBoost به‌واسطه بهینه‌سازی گرادیان تقویتی، توانسته‌اند در مواجهه با داده‌های نامتوازن دقت و حساسیت بالایی ارائه دهند \cite{ref2}.
    \item پژوهش‌های اخیر در استفاده از شبکه‌های عصبی عمیق و یادگیری انتقالی، به بهبود عملکرد سیستم‌های تشخیص باج‌افزار در محیط‌های پیچیده و دنیای واقعی منجر شده‌اند \cite{ref3}.
\end{itemize}
با وجود دستاوردهای حاصل از مطالعات مذکور، چالش‌هایی همچنان در تشخیص دقیق باج‌افزار وجود دارد که در این پژوهش با بهره‌گیری از ترکیب مدل‌ها و مهندسی ویژگی‌های پیشرفته مورد بررسی قرار گرفته است.

\section{بحث}
نتایج به‌دست آمده از ارزیابی مدل‌های پیشنهادی نشان می‌دهد که هر یک از مدل‌ها دارای نقاط قوت و محدودیت‌های خاص خود هستند:
\begin{itemize}
    \item \textbf{رندوم فارست:} به دلیل تفسیرپذیری بالا و پایداری در تمام اجراها، به عنوان مدل مرجع در بسیاری از کاربردها مطرح می‌شود؛ اما در مواجهه با برخی نمونه‌های مرزی حساسیت آن ممکن است کاهش یابد.
    \item \textbf{XGBoost:} با ارائه حساسیت بالا و زمان استنتاج کوتاه، به ویژه برای کاربردهای بلادرنگ، عملکرد قابل قبولی ارائه می‌دهد؛ هرچند تنظیم دقیق پارامترهای آن برای جلوگیری از بیش‌برازش چالشی محسوب می‌شود.
    \item \textbf{شبکه عصبی (MLP):} علی‌رغم نیاز به زمان آموزش بیشتر و پیش‌پردازش‌های پیچیده، در شناسایی الگوهای غیرخطی و نمونه‌های پیچیده عملکرد مناسبی از خود نشان می‌دهد.
\end{itemize}

همچنین تحلیل نتایج نشان می‌دهد:
\begin{enumerate}
    \item استفاده از تکنیک‌های پیش‌پردازش و مهندسی ویژگی‌های ترکیبی، تأثیر مثبتی بر بهبود عملکرد کلی سیستم داشته است.
    \item مدیریت داده‌های نامتوازن، به کمک روش‌هایی نظیر SMOTE و تنظیم وزن کلاس‌ها، نقشی کلیدی در افزایش حساسیت مدل‌ها ایفا نموده است.
    \item وجود نمونه‌های مرزی که الگوهای دسترسی فایل‌های باج‌افزار و نرم‌افزارهای خوش‌خیم را به‌هم نزدیک می‌کند، نیازمند استخراج ویژگی‌های دقیق‌تر و بهبود الگوریتم‌های طبقه‌بندی می‌باشد.
\end{enumerate}

\section{پیشنهادات برای تحقیقات آتی}
بر اساس نتایج به‌دست آمده و مقایسه با مطالعات موجود، پیشنهادات زیر جهت بهبود سیستم تشخیص باج‌افزار و تحقیقات آتی ارائه می‌شود:
\begin{itemize}
    \item \textbf{گسترش پایگاه داده:} گردآوری داده‌های بیشتر و از منابع متنوع، می‌تواند به تعمیم‌پذیری مدل‌ها و کاهش اثر نویز کمک نماید.
    \item \textbf{بهبود استخراج ویژگی‌ها:} استفاده از الگوریتم‌های یادگیری عمیق برای استخراج ویژگی‌های سطح بالا و ویژگی‌های پنهان در داده‌های رفتاری، می‌تواند دقت طبقه‌بندی را افزایش دهد.
    \item \textbf{ترکیب مدل‌ها (Ensemble):} بهره‌گیری از استراتژی‌های ترکیبی جهت تلفیق نقاط قوت مدل‌های مختلف (مثلاً ترکیب رندوم فارست، XGBoost و شبکه عصبی) می‌تواند به عملکرد بهینه‌تری منجر شود.
    \item \textbf{بهینه‌سازی زمان استنتاج:} بررسی روش‌های بهینه‌سازی و کاهش زمان استنتاج، به‌ویژه در کاربردهای بلادرنگ، از اهمیت ویژه‌ای برخوردار است.
    \item \textbf{استفاده از یادگیری انتقالی:} اعمال تکنیک‌های یادگیری انتقالی بر روی مدل‌های پیش‌آموزش‌دیده، می‌تواند در شرایط داده‌های کم و محیط‌های واقعی به بهبود عملکرد کمک نماید.
    \item \textbf{ارزیابی در محیط‌های واقعی:} پیاده‌سازی و آزمایش سیستم در محیط‌های عملی و واقعی، جهت سنجش عملکرد در شرایط متغیر و پیچیده دنیای واقعی، توصیه می‌شود.
\end{itemize}

\section{نتیجه‌گیری فصل}
در این فصل، با مرور ادبیات مرتبط، به بررسی نقاط قوت و ضعف سیستم‌های تشخیص باج‌افزار پرداخته شد و نتایج به‌دست آمده مورد بحث قرار گرفت. همچنین، پیشنهاداتی جهت بهبود سیستم و جهت تحقیقات آینده ارائه گردید. این پیشنهادات می‌تواند به عنوان مبنایی برای توسعه روش‌های پیشرفته‌تر در زمینه امنیت سایبری و تشخیص تهدیدات نوین مورد استفاده قرار گیرد.
