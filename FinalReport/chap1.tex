\chapter{مقدمه}

\section{پیشینه تحقیق}
با گسترش فناوری‌های دیجیتال، تهدیدات سایبری به‌ویژه بدافزارها به چالشی جهانی تبدیل شده‌اند. در این میان، باج‌افزارها به‌عنوان زیرمجموعه‌ای خطرناک از بدافزارها با رمزگذاری داده‌های قربانی و اخاذی مالی، یکی از مخرب‌ترین تهدیدهای امنیتی دهه‌ی اخیر به‌شمار می‌روند. طبق گزارش \cite{cybertalk2023}، حملات باج‌افزاری در سال ۲۰۲۳ نسبت به سال قبل ۶۷٪ افزایش یافته و میانگین خسارت هر حمله به ۱.۸۵ میلیون دلار رسیده است.

روش‌های سنتی مبتنی بر امضا (\lr{Signature-based}) به‌علت ناتوانی در شناسایی گونه‌های جدید باج‌افزارها، کارایی محدودی دارند. در مقابل، رویکردهای مبتنی بر یادگیری ماشین با تحلیل الگوهای رفتاری و متادیتای فایل‌ها، امکان شناسایی حملات ناشناخته را فراهم می‌کنند. مطالعات اخیر \cite{li2022} نشان می‌دهد استفاده‌ی ترکیبی از ویژگی‌های استاتیک و دینامیک می‌تواند دقت تشخیص را تا ۹۸٪ بهبود بخشد. با این حال، چالش‌هایی مانند عدم تعادل کلاس‌ها، نویز در داده‌ها و انتخاب بهینه‌ی ویژگی‌ها همچنان وجود دارد.

در پژوهش‌های جدیدتر، رویکردهای ترکیبی و مدل‌های پیچیده‌تری همچون شبکه‌های عصبی عمیق، گرادیان بوست (\lr{Gradient Boosted Trees}) و \lr{XGBoost} توانسته‌اند دقت قابل توجهی را در تشخیص باج‌افزارها ارائه دهند.

مطابق جدول گزارش‌شده در پژوهش \lr{RDPM} \cite{singh2022ransomwaredetectionusingprocess} (جدول \ref{tab:rdpm_results})، برخی روش‌های متداول در تشخیص باج‌افزار و دقت آن‌ها به‌صورت زیر است:


با توجه به تصاویر ارسال‌شده، در این پژوهش، با افزودن روش کراس‌ولیدیشن ۵-فولدی به مدل شبکه‌ی عصبی (\lr{Neural Network})، دقت مدل از حدود ۵۰٪ (در آزمایش اولیه با داده‌های نامتوازن) به ۸۹--۹۶٪ در سناریوهای مختلف افزایش یافته است. این پیشرفت نشان می‌دهد با تنظیم مناسب معماری شبکه، توازن داده‌ها (به‌کمک روش‌هایی مانند \lr{SMOTE}) و انتخاب دقیق ویژگی‌ها، می‌توان در رقابت با روش‌های گزارش‌شده در \lr{RDPM} عمل کرد یا حتی در برخی جنبه‌ها بهبود داشت.

\section{اهداف و دستاوردهای تحقیق}
اهداف اصلی این پژوهش عبارتند از:
\begin{itemize}
  \item طراحی چارچوبی جامع برای تشخیص باج‌افزارها با ترکیب یادگیری ماشین و مهندسی ویژگی‌ها
  \item توسعه‌ی ویژگی‌های نوین مبتنی بر الگوهای دسترسی فایل (\lr{rwx, rwc, ...})
  \item مقابله با مشکل عدم تعادل داده‌ها از طریق تکنیک‌های ترکیبی نمونه‌برداری
  \item مقایسه‌ی عملکرد مدل‌های کلاسیک، مدل‌های بوستینگ و شبکه‌های عصبی در این حوزه
  \item پیاده‌سازی و ارزیابی مدل شبکه‌ی عصبی با روش کراس‌ولیدیشن ۵-فولدی
\end{itemize}

دستاوردهای کلیدی این پژوهش:
\begin{itemize}
  \item دستیابی به \lr{AUC-ROC} برابر ۹۸٪ با مدل \lr{Random Forest} در مجموعه داده‌ی متوازن
  \item ارتقای دقت مدل شبکه‌ی عصبی از ۵۰٪ به حدود ۹۰--۹۵٪ با افزودن کراس‌ولیدیشن ۵-فولدی (با توجه به نتایج مشاهده‌شده در تصاویر)
  \item کاهش ۴۰٪ نرخ مثبت کاذب نسبت به روش‌های مبتنی بر امضا
  \item طراحی ۵ ویژگی جدید ترکیبی (مانند \lr{Complexity Score})
  \item ایجاد مجموعه داده‌ی متوازن با نسبت ۱:۱ از نمونه‌های سالم و آلوده
\end{itemize}

\section{ساختار گزارش}
این گزارش در ۶ فصل سازماندهی شده است:
\begin{enumerate}
  \item \textbf{فصل ۲: پیش‌پردازش داده} - بررسی چالش‌های داده، روش‌های پاکسازی و مهندسی ویژگی‌ها
  \item \textbf{فصل ۳: روش‌شناسی} - تشریح معماری مدل‌ها، معیارهای ارزیابی و منطق پیاده‌سازی
  \item \textbf{فصل ۴: ارزیابی نتایج} - تحلیل خروجی مدل‌ها، نمودارهای \lr{ROC} و \lr{PR}، ماتریس سردرگمی و مقایسه با کارهای مرتبط
  \item \textbf{فصل ۵: بحث، پیشنهادات و مرور ادبیات} - بررسی محدودیت‌ها و مسیرهای توسعه‌ی آینده
  \item \textbf{فصل ۶: نتیجه‌گیری و پیشنهادات جهت تحقیقات آتی} - جمع‌بندی یافته‌های کلیدی و کاربردهای عملی
\end{enumerate}


