%*************************************************
% In this file the abstract is typeset.
% Make changes accordingly.
%*************************************************

\addcontentsline{toc}{section}{چکیده}
\newgeometry{left=2.5cm,right=3cm,top=3cm,bottom=2.5cm,includehead=false,headsep=1cm,footnotesep=.5cm}
\setcounter{page}{1}
\pagenumbering{arabic}
\thispagestyle{empty}

~\vfill

\subsection*{چکیده}
\begin{small}
\baselineskip=0.7cm
در این پروژه به بررسی روش‌های تشخیص و طبقه‌بندی باج‌افزار می‌پردازیم. با گسترش روزافزون حملات سایبری، باج‌افزارها به عنوان یکی از مخرب‌ترین تهدیدهای امنیتی، خسارات مالی و عملیاتی قابل توجهی به سازمان‌ها و افراد وارد می‌کنند. گزارش‌های اخیر نشان می‌دهند که در سال ۲۰۲۳، بیش از ۷۲٪ سازمان‌ها حداقل یک بار هدف حملات باج‌افزاری قرار گرفته‌اند، که این امر لزوم توسعه سیستم‌های تشخیص هوشمند و کارآمد را بیش از پیش آشکار می‌سازد. اغلب روش‌های استفاده شده توسط من از الگوریتم‌های یادگیری ماشین و یادگیری عمیق برای شناسایی و مقابله با حملات باج‌افزاری استفاده می‌کنند. ابتدا داده‌های حاصل از رفتارهای نرم‌افزارها جمع‌آوری و تحلیل می‌شوند تا ویژگی‌های کلیدی و الگوهای رفتاری مرتبط با باج‌افزارها شناسایی و استخراج شوند. این ویژگی‌ها سپس به مدل‌های مختلف یادگیری ماشین و عمیق داده می‌شوند تا برای تشخیص و طبقه‌بندی باج‌افزارها آموزش ببینند. در این فرآیند، یک یا چند نمونه از این مدل‌ها پیاده‌سازی شده و سپس کارایی و دقت هر مدل با استفاده از معیارهای ارزیابی مختلف سنجیده شده و مدل‌های منتخب برای تشخیص دقیق‌تر انتخاب می‌شوند. این سیستم می‌تواند در محیط‌های مختلف برای جلوگیری از نفوذ و گسترش باج‌افزار مورد استفاده قرار گیرد.

\noindent\textbf{کلمات کلیدی:} تشخیص باج‌افزار، امنیت سایبری، درخت تصمیم، جنگل تصادفی، پرسپترون چند لایه
\end{small}