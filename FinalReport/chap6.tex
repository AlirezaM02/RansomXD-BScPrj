% Chapter 6
\chapter{نتیجه‌گیری و پیشنهادات جهت تحقیقات آتی}

\section{خلاصه دستاوردها}
در این پژوهش با بهره‌گیری از روش‌های پیش‌پردازش داده، مهندسی ویژگی‌های ترکیبی و استفاده از مدل‌های یادگیری ماشین پیشرفته، یک سیستم تشخیص باج‌افزار توسعه داده شد که در شرایط داده‌های نامتوازن و نویزی عملکرد قابل قبولی از خود نشان داد. دستاوردهای کلیدی عبارتند از:
\begin{itemize}
    \item طراحی چارچوبی جامع برای پاکسازی، نرمال‌سازی و استخراج ویژگی‌های کلیدی از داده‌های اولیه.
    \item پیاده‌سازی و بهینه‌سازی مدل‌های رندوم فارست، XGBoost و شبکه عصبی (MLP) با استفاده از استراتژی‌های اعتبارسنجی متقابل و تنظیم دقیق هایپرپارامترها.
    \item ارزیابی عملکرد مدل‌ها از طریق معیارهای متعددی نظیر دقت، حساسیت، F1-Score، ROC AUC و Average Precision که نشان‌دهنده قابلیت استفاده عملی سیستم در تشخیص باج‌افزار می‌باشد.
\end{itemize}

\section{نتیجه‌گیری نهایی}
بر اساس نتایج حاصل از آزمایشات، می‌توان نتیجه گرفت که:
\begin{itemize}
    \item \textbf{رندوم فارست} به دلیل تفسیرپذیری بالا و پایداری عملکرد در شرایط مختلف، به عنوان مدل مرجع جهت کاربردهای غیر بلادرنگ مورد استفاده قرار گیرد.
    \item \textbf{XGBoost} با ارائه حساسیت بسیار بالا و زمان استنتاج کم، گزینه مناسبی برای کاربردهای بلادرنگ محسوب می‌شود.
    \item \textbf{شبکه عصبی (MLP)} علی‌رغم نیاز به پیش‌پردازش‌های بیشتر و زمان آموزش طولانی‌تر، در شناسایی الگوهای غیرخطی و پیچیده عملکرد مناسبی از خود ارائه می‌دهد.
\end{itemize}
این یافته‌ها نشان می‌دهد که انتخاب مدل نهایی باید با توجه به نیازهای کاربردی و شرایط محیطی مورد استفاده قرار گیرد.

\section{پیشنهادات جهت تحقیقات آتی}
با توجه به نتایج به‌دست آمده و چالش‌های موجود، پیشنهادات زیر جهت تحقیقات آتی ارائه می‌شود:
\begin{itemize}
    \item \textbf{افزایش تنوع و حجم داده‌ها:} گردآوری داده‌های بیشتر از منابع متنوع، بهبود تعمیم‌پذیری مدل‌ها و کاهش اثر نویز در داده‌های واقعی را به همراه خواهد داشت.
    \item \textbf{بهبود استخراج ویژگی:} استفاده از الگوریتم‌های یادگیری عمیق برای استخراج ویژگی‌های سطح بالا و کشف الگوهای پنهان، می‌تواند دقت سیستم تشخیص را افزایش دهد.
    \item \textbf{ترکیب مدل‌ها (Ensemble):} تلفیق نقاط قوت مدل‌های مختلف (مثلاً از طریق روش‌های Ensemble مانند Bagging و Boosting) می‌تواند عملکرد کلی سیستم را بهبود بخشد.
    \item \textbf{بهینه‌سازی زمان استنتاج:} توسعه الگوریتم‌های سبک‌تر یا استفاده از تکنیک‌های بهینه‌سازی سخت‌افزاری جهت کاهش زمان استنتاج، به‌ویژه در کاربردهای بلادرنگ، از اهمیت ویژه‌ای برخوردار است.
    \item \textbf{ارزیابی در محیط‌های واقعی:} آزمایش سیستم در محیط‌های عملی و واقعی به منظور ارزیابی عملکرد در شرایط متغیر و پیچیده دنیای واقعی، می‌تواند بینش بهتری نسبت به قابلیت اجرایی سیستم ارائه دهد.
    \item \textbf{استفاده از یادگیری انتقالی:} بهره‌گیری از مدل‌های پیش‌آموزش‌دیده و تطبیق آن‌ها با داده‌های جدید، می‌تواند در شرایط داده‌های محدود و متغیر به بهبود عملکرد کمک نماید.
\end{itemize}

\section{جمع‌بندی}
این پژوهش با ارائه یک سیستم جامع تشخیص باج‌افزار از طریق به‌کارگیری روش‌های نوین پیش‌پردازش، مهندسی ویژگی و مدل‌های پیشرفته یادگیری ماشین، گامی مؤثر در جهت مقابله با تهدیدات سایبری محسوب می‌شود. پیشنهادات مطرح‌شده می‌تواند راهگشای تحقیقات آینده در زمینه امنیت سایبری و توسعه سیستم‌های تشخیص تهدیدات باشد.
